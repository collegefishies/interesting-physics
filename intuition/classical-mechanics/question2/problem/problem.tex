\subsection{Why does anything move at all?}
Consider a student tossing up a ball, Newton's third law claims that for every action there is an equal and opposite reaction. That is, for bodies $A$ and $B$, if $A$ exerts a force on $B$ then $B$ exerts a force on $A$ equal in magnitude, opposite and direction, and of the same physical nature\footnote{Gravitational/gravitational, electrical/electrical, but never anything like gravitational/electrical.}. 

If this is the case, how is it that a student can even toss up a ball? The ball is always pushing her arm down with equal force that she's pushing the ball with so how is it that it can move up?